\documentclass[]{article}
\usepackage{lmodern}
\usepackage{amssymb,amsmath}
\usepackage{ifxetex,ifluatex}
\usepackage{fixltx2e} % provides \textsubscript
\ifnum 0\ifxetex 1\fi\ifluatex 1\fi=0 % if pdftex
  \usepackage[T1]{fontenc}
  \usepackage[utf8]{inputenc}
\else % if luatex or xelatex
  \ifxetex
    \usepackage{mathspec}
  \else
    \usepackage{fontspec}
  \fi
  \defaultfontfeatures{Ligatures=TeX,Scale=MatchLowercase}
\fi
% use upquote if available, for straight quotes in verbatim environments
\IfFileExists{upquote.sty}{\usepackage{upquote}}{}
% use microtype if available
\IfFileExists{microtype.sty}{%
\usepackage{microtype}
\UseMicrotypeSet[protrusion]{basicmath} % disable protrusion for tt fonts
}{}
\usepackage[margin=1in]{geometry}
\usepackage{hyperref}
\hypersetup{unicode=true,
            pdftitle={STA3030 Project 2},
            pdfauthor={Kiara Beilinsohn},
            pdfborder={0 0 0},
            breaklinks=true}
\urlstyle{same}  % don't use monospace font for urls
\usepackage{graphicx,grffile}
\makeatletter
\def\maxwidth{\ifdim\Gin@nat@width>\linewidth\linewidth\else\Gin@nat@width\fi}
\def\maxheight{\ifdim\Gin@nat@height>\textheight\textheight\else\Gin@nat@height\fi}
\makeatother
% Scale images if necessary, so that they will not overflow the page
% margins by default, and it is still possible to overwrite the defaults
% using explicit options in \includegraphics[width, height, ...]{}
\setkeys{Gin}{width=\maxwidth,height=\maxheight,keepaspectratio}
\IfFileExists{parskip.sty}{%
\usepackage{parskip}
}{% else
\setlength{\parindent}{0pt}
\setlength{\parskip}{6pt plus 2pt minus 1pt}
}
\setlength{\emergencystretch}{3em}  % prevent overfull lines
\providecommand{\tightlist}{%
  \setlength{\itemsep}{0pt}\setlength{\parskip}{0pt}}
\setcounter{secnumdepth}{0}
% Redefines (sub)paragraphs to behave more like sections
\ifx\paragraph\undefined\else
\let\oldparagraph\paragraph
\renewcommand{\paragraph}[1]{\oldparagraph{#1}\mbox{}}
\fi
\ifx\subparagraph\undefined\else
\let\oldsubparagraph\subparagraph
\renewcommand{\subparagraph}[1]{\oldsubparagraph{#1}\mbox{}}
\fi

%%% Use protect on footnotes to avoid problems with footnotes in titles
\let\rmarkdownfootnote\footnote%
\def\footnote{\protect\rmarkdownfootnote}

%%% Change title format to be more compact
\usepackage{titling}

% Create subtitle command for use in maketitle
\newcommand{\subtitle}[1]{
  \posttitle{
    \begin{center}\large#1\end{center}
    }
}

\setlength{\droptitle}{-2em}
  \title{STA3030 Project 2}
  \pretitle{\vspace{\droptitle}\centering\huge}
  \posttitle{\par}
  \author{Kiara Beilinsohn}
  \preauthor{\centering\large\emph}
  \postauthor{\par}
  \predate{\centering\large\emph}
  \postdate{\par}
  \date{13 March 2019}


\begin{document}
\maketitle

\subsection{Introduction}\label{introduction}

The ojective for this assignemnt is to analyse the data from two sample
datasets. The datasets C1 and D2 were utiliesed. The first segemnt of
data for this project was obtained from 30 different people who filled
in a form before and after it was changed (paired t test), the times
were recorded. The data was collected by the internal revenue service as
they were looking for ways to improve the wording and format of its tax
return forms.The aim for this study is to test if there is significant
differences in the completion time. The second segment of this data was
obtained from a study perfomed by the Columbia University with regards
to two different departmends and the amount of times lectures said
``uh'' and ``ah'' in lectures to fill the gap between words. 50
observations were taken from each department. This assignment focuses on
testing weather there are significant differences between each of the
groups in the different data sets. It will focus on the differences in
means and variences, using a variety of different methods; namely
Bootstrap resampling and Standard Normal theory. The results will also
provide us with a condifence interval for the differnece between the two
means.

The data provided is as follows:\\
form1 = 23, 59, 68, 122, 74, 90, 70, 87, 155, 120, 124, 103, 54, 90,
124, 80, 69, 123, 76, 71, 94, 167, 69, 105, 98, 73, 79, 61, 121, 56\\
form2 = 88, 114, 81, 41, 108, 92, 52, 54, 103, 50, 135, 76, 143, 124,
151, 96, 76, 128, 60, 127, 109, 122, 88, 109, 90, 56, 105, 64, 127, 104

\subsection{Question 1}\label{question-1}

Question 1, involves a bootstrapping method. I created 5000 bootstrapped
statistics by replicating the original sampling process. I resample by
placing all observations from the original sample into a hat, then
drawing the wanted observations from the hat with replacement to form a
bootstrap sample. A mean is then calculated for each bootstrap sample,
the first 5 of these values can be seen below. I then subtract these
means from the two bootstrap samples of the different original samples
to obtain a specific bootstrap statistic, which I store in an array.
This is done 5000 times.

The hypothesis for this test are::\\
\(H_{0} : \mu_n = \mu_p\) (There is no difference in the population
means)\\
\(H_{1} : \mu_n \ne \mu_p\) (The form1 population mean is different to
the form2 population mean)

I let \(\bar{x_b}\) denote the bootstrap statistic and \(\bar{x_d}\)
denote the difference between the means of the original samples.
Therefore the sampling error is \(\bar{x_b}-\bar{x_d}\).Then the
observed discrepencies are calculated which is \(\bar{x_d}-0\) under the
null hypothesis. The p-value is then calculated using
\(p=Pr(\bar{x_b}-\bar{x_d}\geq \bar{x_d}-0)=Pr(\bar{x_b}\geq2\bar{x_d}-0)\).
The p-value through bootstrapping is 0.1552.

The p-value is defined as the probability of observing a similar or more
extreme result, assuming the null hypothesis is true. With the
calculated pvalue, it can be concluded that there is strong evidence to
fail to reject the null hypothesis and conclude that there is no
difference in the performance on the two forms and that the population
means are equal.

\begin{verbatim}
## [1] "First five values:"
\end{verbatim}

\begin{verbatim}
## [1] 2.233333
## [1] -22.63333
## [1] 7.266667
## [1] -6.633333
## [1] -5.666667
\end{verbatim}

\begin{verbatim}
## [1] "The pvalue is:"
\end{verbatim}

\begin{verbatim}
## [1] 0.1552
\end{verbatim}

In addition the analysis of the bootstrap Mean through a histogram
displays a symetrical distribution which looks approxematly like the
Normal distribution, this can be seen below:

\includegraphics{STA3030Project2_files/figure-latex/unnamed-chunk-4-1.pdf}

\subsection{Question 2}\label{question-2}

Question 2 involves calculating the 95\% confidence interval for the
difference in means. It makes use of the bootstrap principle:
(\(\bar{x_b}-\bar{x}\approx\bar{x}-\mu)\)) as it uses bootstrapping
theory.

\begin{verbatim}
## [1] "The 95% confidence intervals for the means is"
\end{verbatim}

\begin{verbatim}
## [1] -26.333   4.533
\end{verbatim}

\subsection{Question 3}\label{question-3}

The data used for Question 3 is as follows:

dept1 = 4, 9, 8, 7, 4, 4, 8, 7, 9, 0, 7, 4, 7, 8, 4, 4, 7, 5, 0, 3, 3,
1, 3, 3, 4, 7, 10, 8, 7, 5, 1, 5, 3, 4, 5, 10, 5, 8, 4, 4, 4, 9, 5, 9,
5, 7, 3, 8, 10, 4\\
dept2 = 5, 4, 9, 5, 5, 5, 6, 4, 5, 5, 4, 0, 6, 7, 4, 3, 3, 6, 7, 4, 5,
1, 8, 9, 5, 9, 4, 7, 0, 9, 5, 6, 4, 5, 5, 6, 5, 5, 3, 4, 5, 6, 4, 3, 8,
3, 5, 7, 8, 8

\section{Using Bootstrapping:}\label{using-bootstrapping}

\begin{verbatim}
## [1] "The F statistic is"
\end{verbatim}

\begin{verbatim}
## [1] 1.543412
\end{verbatim}

\begin{verbatim}
## [1] "The pvalue is"
\end{verbatim}

\begin{verbatim}
## [1] 0.5054
\end{verbatim}

An anyalysis of the bootstrap variences plotted as a histogram
demonstrated a relativly F distributed distribution as seen below:

\includegraphics{STA3030Project2_files/figure-latex/unnamed-chunk-8-1.pdf}

\subsection{Question 4}\label{question-4}

\section{Using Normal theory}\label{using-normal-theory}

\begin{verbatim}
## 
##  F test to compare two variances
## 
## data:  dept2 and dept1
## F = 0.64792, num df = 49, denom df = 49, p-value = 0.1322
## alternative hypothesis: true ratio of variances is not equal to 1
## 95 percent confidence interval:
##  0.3676765 1.1417489
## sample estimates:
## ratio of variances 
##          0.6479153
\end{verbatim}

\begin{verbatim}
## [1] "The pvalue is:"
\end{verbatim}

\begin{verbatim}
## [1] 0.1322171
\end{verbatim}

A comparision of the results from Q3 Bootstrapping method and Q4 normal
theroy method shows: \#\# Question 5

An interpretation of results can be seen as follows:

\subsection{Conclusion}\label{conclusion}


\end{document}
